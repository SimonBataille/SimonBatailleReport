% Chapter 4

\chapter{Internship Subject} % Main chapter title

\label{Chapter4} % For referencing the chapter elsewhere, use \ref{Chapter4}

%----------------------------------------------------------------------------------------

% Define some commands to keep the formatting separated from the content
%\newcommand{\keyword}[1]{\textbf{#1}}
%\newcommand{\tabhead}[1]{\textbf{#1}}
%\newcommand{\code}[1]{\texttt{#1}}
%\newcommand{\file}[1]{\texttt{\bfseries#1}}
%\newcommand{\option}[1]{\texttt{\itshape#1}}

%----------------------------------------------------------------------------------------

\section{Problematics}

As I said in the previous parts, \iBubble{} includes a visual tracking system to follow divers underwater. This system currently  uses \emph{openCV} libraries at a rate of \textbf{30 fps}. This speed can be improved using CUDA with a graphic chip. These features are only available on \keyword{BtB} versions of \iBubble fitted with \emph{Nvidia's Jetson embedded board}.

Indeed, \keyword{BtC} versions of the drone uses \rasp{} in spite of the current Nvidia Graphics Card wich means that CUDA is no longer available.
La pi a un GPU mais il ne supporte pas CUDA (c'est un videocore IV) et du coup, il est inutilisé ==> le tracking tourne sur CPU ==> on tombe à 7fps.La partie qui prend le plus de temps CPU est quelque chose qui s'appelle le "flux optique" qui est un algo de suivi fondamental en Computer Vision et qui utilise la descente de gradient pour obtenir le déplacement de points d'intérêt. Du coup, si on veut utiliser la puissance GPU on est obligé de réimplémenter cet algo sur le GPU de la raspberry pi.Donc non seulement Simon fait la descente de gradient lui-meme, mais il la fait en assembleur. Tout ça c'est dans le cadre du stage de Camille \& Simon

%----------------------------------------------------------------------------------------

\section{Optical Flow}

%----------------------------------------------------------------------------------------

\section{Lucas-Kanade algorithm}

%----------------------------------------------------------------------------------------

\section{Implementation on \vc}

%----------------------------------------------------------------------------------------
