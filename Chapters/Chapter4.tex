% Conclusion

\chapter{Conclusion} % Main chapter title

\label{Conclusion} % For referencing the chapter elsewhere, use \ref{Conclusion}

%----------------------------------------------------------------------------------------

% Define some commands to keep the formatting separated from the content
%\newcommand{\keyword}[1]{\textbf{#1}}
%\newcommand{\tabhead}[1]{\textbf{#1}}
%\newcommand{\code}[1]{\texttt{#1}}
%\newcommand{\code}[1]{\texttt{\hl{#1}}}
%\newcommand{\file}[1]{\texttt{\bfseries#1}}
%\newcommand{\option}[1]{\texttt{\itshape#1}}
%\newcommand{\iBubble}{\textsc{iBubble}}
%\newcommand{\rasp}{\textsc{Raspberry Pi}}
%\newcommand{\vc}{\textsc{VideoCore iv 3D}}
%\newcommand{\cpu}{\textsc{arm cpu}}
%\newcommand{\bcm}{\textsc{bcm2837}}
%\newcommand{\qpu}{\textsc{qpu}}
%\newcommand{\flow}{\textsc{optical flow}}
%\newcommand{\feat}{\textsc{feature}}
%\newcommand{\api}{\textsc{api}}
%\newcommand{\ram}{\textsc{shared ram}}
%\newcommand{\mail}{\textsc{mailbox}}
%\newcommand{\uni}{\textsc{uniform}}

%----------------------------------------------------------------------------------------

\section{Results}

To test this \api{} I compute the \flow{} first using \code{calcOpticalFlowPyrLK()} and then using \code{compute\_lk\_gpu()}: listing~\ref{mainLst}. Running the project with \code{sudo ./optical\_flow\_internship /home/pi/video/video\_out.avi} generates a directory containing three files:
\begin{itemize}
	\item \file{video\_out\_init\_features.jpg}: an image containing the initial \feat{}s positions - Figure~\ref{initFeaturesFig}
	\item \file{video\_out\_opt\_flow\_vectors.avi}: a video where \flow{} between each consecutive frames is representing by \textcolor{blue}{blue vectors} - Figure~\ref{opticalFlowFig}
	\item \file{video\_out\_vectors.json}: a \file{.json} file containing all the 30 $(d_{x},d_{y})$ values for each frame of the whole \emph{video\_sample}
\end{itemize}

Moreover the terminal displays time processing for each frame of the video. So I am able to compare time when using this two functions.


\subsection{Time Comparison}

\lstset{language=bash,caption={\code{calcOpticalFlowPyrLK} terminal output},basicstyle=\tiny,label=}
\begin{lstlisting}
mean time per frame: 0.00538982 for 24 frames.
JSON file of vectors of each features at each frames can be found at: /home/pi/video/video_out/video_out_vectors.json

Video with vector representing optical flow has been written at : /home/pi/video/video_out/video_out_opt_flow_vectors.avi

**********************************************************************
**********************************************************************
All videos / frames of: /home/pi/video have been processed !
**********************************************************************
\end{lstlisting}


\lstset{language=bash,caption={\code{calcOpticalFlowPyrLK} terminal output},basicstyle=\tiny,label=}
\begin{lstlisting}
mean time per frame: 0.0119438 for 24 frames.
JSON file of vectors of each features at each frames can be found at: /home/pi/video/video_out/video_out_vectors.json

Video with vector representing optical flow has been written at : /home/pi/video/video_out/video_out_opt_flow_vectors.avi

**********************************************************************
**********************************************************************
All videos / frames of: /home/pi/video have been processed !
**********************************************************************
\end{lstlisting}



\subsection{Values Comparison}

\noindent\begin{minipage}{.45\textwidth}
\begin{lstlisting}[caption={\code{calcOpticalFlowPyrLK} values},frame=tlrb,basicstyle=\tiny]{Name}
"number": "9",
"vectors": {
	"feature 1": "[0.00437927, 1.01498]",
	"feature 2": "[0.0429688, 0.984863]",
	"feature 3": "[-0.00196838, 0.985191]",
	"feature 4": "[0.013916, 0.999466]",
	"feature 5": "[-0.00466919, 1.02369]",
	"feature 6": "[0.00923157, 1.59529]",
	"feature 7": "[0.136581, 1.39946]",
	"feature 8": "[0.320084, 1.5918]",
	"feature 9": "[0.0149536, 1.03059]",
	"feature 10": "[-0.0966644, 1.72192]",
	"feature 11": "[0.00205994, 0.997665]",
	"feature 12": "[0.175995, 1.14218]",
	"feature 13": "[-0.0109406, 1.09407]",
	"feature 14": "[0.0992737, 1.05608]",
	"feature 15": "[0.205734, 1.56096]",
	"feature 16": "[0.0429688, 1.02802]",
	"feature 17": "[0.243973, 1.51352]",
	"feature 18": "[0.0153656, 1.83011]",
	"feature 19": "[0.0600433, 1.05858]",
	"feature 20": "[-0.000396729, 1.00053]",
	"feature 21": "[0.0501404, 1.38897]",
	"feature 22": "[0.12709, 1.79247]",
	"feature 23": "[0.289734, 2.45174]",
	"feature 24": "[0.0285187, 0.95536]",
	"feature 25": "[0.161499, 1.30433]",
	"feature 26": "[0.12973, 1.23988]",
	"feature 27": "[0.108826, 0.860413]",
	"feature 28": "[0, 0]",
	"feature 29": "[0, 0]",
	"feature 30": "[0, 0]"
}
\end{lstlisting}
\end{minipage}\hfill
\begin{minipage}{.45\textwidth}
	\begin{lstlisting}[caption={\code{compute\_lk\_gpu} values},frame=tlrb,basicstyle=\tiny]{Name}
"number": "9",
"vectors": {
	"feature 1": "[0.0307007, 1.05986]",
	"feature 2": "[0.0518036, 1.08145]",
	"feature 3": "[0.00424194, 1.06021]",
	"feature 4": "[0.015625, 1.01106]",
	"feature 5": "[0.00361633, 0.998215]",
	"feature 6": "[-0.0257874, 1.5815]",
	"feature 7": "[0.249847, 1.38773]",
	"feature 8": "[0.265816, 1.47562]",
	"feature 9": "[0.00706482, 0.996964]",
	"feature 10": "[-0.0304565, 1.66684]",
	"feature 11": "[0.0141449, 0.985558]",
	"feature 12": "[0.0645599, 1.18579]",
	"feature 13": "[0.0774689, 1.00171]",
	"feature 14": "[0.0397186, 1.09403]",
	"feature 15": "[-0.0235443, 1.05531]",
	"feature 16": "[0.0857086, 1.01047]",
	"feature 17": "[0.14328, 1.62175]",
	"feature 18": "[-0.0360718, 1.74749]",
	"feature 19": "[0.0405731, 1.0332]",
	"feature 20": "[-0.0058136, 0.991623]",
	"feature 21": "[0.0559845, 1.42413]",
	"feature 22": "[0.00138855, 1.08983]",
	"feature 23": "[0.00791931, 0.980244]",
	"feature 24": "[-0.0236969, 1.04613]",
	"feature 25": "[-0.0176392, 1.65675]",
	"feature 26": "[0.0956726, 1.2908]",
	"feature 27": "[0.0596008, 1.01413]",
	"feature 28": "[0.00686646, 0.994537]",
	"feature 29": "[-0.0133667, 1.12775]",
	"feature 30": "[-0.00701904, 2.02641]"
}
\end{lstlisting}
\end{minipage}


%----------------------------------------------------------------------------------------

\section{Possibilities}

As of the writing of this report, a part of the Lucas-Kanade method was still in development. This part called \emph{Gaussian Pyramids} is a way to estimate large displacement - large values of $(d_{x},d_{y})$ - between two frames.

Inputs are still \textcolor{blue}{firstFrame}, \textcolor{red}{secondFrame} and \file{int} $featuresArray[2][30]$. The \emph{Gaussian Pyramids} algorithm is invoked before displacement computation section~\ref{dispComp}. This algorithm provides initial approached values of the $(d_{x},d_{y})$ for each feature. Those values are injected on the first iteration of displacement computation.

\subsection{Pyramid Algorithm}

\subsubsection{First convolution}

\noindent\begin{minipage}{.3\textwidth}
\[
\begin{bmatrix}

firstPixel_{0,0} & firstPixel_{0,1} & \ldots & \ldots & firstPixel_{0,359}\\

firstPixel_{1,0} & \ddots & \ldots & \ldots & firstPixel_{1,359}\\

\vdots & \ldots & \ddots & \ldots & \vdots\\

\vdots & \ldots & \ldots & \ddots & \vdots\\

firstPixel_{239,0} & \ldots & \ldots  & \ldots & firstPixel_{239,359}

\end{bmatrix}_{240\times 360}
\]
\end{minipage}\hfill
\begin{minipage}{.3\textwidth}
\[
\begin{bmatrix}

1 & 2 & 1\\

2 & 4 & 2\\

1 & 2 & 1\\

\end{bmatrix}_{3\times 3 kernel}
\]
\end{minipage}


\subsubsection{Second convolution}

\noindent\begin{minipage}{.3\textwidth}
\[
\begin{bmatrix}

secondPixel_{0,0} & secondPixel_{0,1} & \ldots & \ldots & secondPixel_{0,357}\\

secondPixel_{1,0} & \ddots & \ldots & \ldots & secondPixel_{1,357}\\

\vdots & \ldots & \ddots & \ldots & \vdots\\

\vdots & \ldots & \ldots & \ddots & \vdots\\

secondPixel_{237,0} & \ldots & \ldots  & \ldots & secondPixel_{237,357}

\end{bmatrix}_{238\times 358}
\]
\end{minipage}\hfill
\begin{minipage}{.3\textwidth}
\[
\begin{bmatrix}

1 & 2 & 1\\

2 & 4 & 2\\

1 & 2 & 1\\

\end{bmatrix}_{3\times 3 kernel}
\]
\end{minipage}


\subsubsection{Extraction}

\[
\begin{bmatrix}

thirdPixel_{0,0} & thirdPixel_{0,1} & \ldots & \ldots & thirdPixel_{0,355}\\

thirdPixel_{1,0} & \ddots & \ldots & \ldots & thirdPixel_{1,355}\\

\vdots & \ldots & \ddots & \ldots & \vdots\\

\vdots & \ldots & \ldots & \ddots & \vdots\\

thirdPixel_{235,0} & \ldots & \ldots  & \ldots & thirdPixel_{117,355}

\end{bmatrix}_{236\times 356}
\]



\[
\begin{bmatrix}

thirdPixel_{0,0} & thirdPixel_{0,2} & \ldots & \ldots & thirdPixel_{0,354}\\

thirdPixel_{2,0} & \ddots & \ldots & \ldots & thirdPixel_{2,354}\\

\vdots & \ldots & \ddots & \ldots & \vdots\\

\vdots & \ldots & \ldots & \ddots & \vdots\\

thirdPixel_{234,0} & \ldots & \ldots  & \ldots & thirdPixel_{234,354}

\end{bmatrix}_{118\times 178}
\]

\subsubsection{Optical flow algorithm}

Once we get the $118\times 178$ matrix from first and second frame, we apply the \flow{} the same way as in ~\ref{Chapter3} with all values of features posions divided by two.

We get 30 $d_{x},d_{y}$ approached values of the features displacement and we can start the \flow{} algorithm for the $240\times 360$ first and second frame to get the exact values of the displacement.
%----------------------------------------------------------------------------------------

