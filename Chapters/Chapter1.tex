% Introduction

\chapter{Introduction} % Main chapter title

\label{Introduction} % For referencing the chapter elsewhere, use \ref{Introduction}

%----------------------------------------------------------------------------------------

% Define some commands to keep the formatting separated from the content
\newcommand{\keyword}[1]{\textbf{#1}}
\newcommand{\tabhead}[1]{\textbf{#1}}
\newcommand{\code}[1]{\texttt{#1}}
\newcommand{\file}[1]{\texttt{\bfseries#1}}
\newcommand{\option}[1]{\texttt{\itshape#1}}
\newcommand{\iBubble}{\textcolor{mdtRed}{\textsc{iBubble}}}
\newcommand{\rasp}{\textcolor{mdtRed}{\textsc{Raspberry Pi}}}
\newcommand{\vc}{\textcolor{mdtRed}{\textsc{VideoCore IV 3D GPU}}}

%----------------------------------------------------------------------------------------

\section{\textsc{Notilo+} \& \textsc{iBubble}}

\footnote{\groupname{} website: \url{https://ibubble.camera/}}\groupname{} is a young and innovative start-up based in Lyon. It was created in 2016 by Nicolas \textsc{Gambini} and Benjamin \textsc{Valtin} and early funded through crowd-funding during the 2016 year.

Currently about 20 people are working at \groupname{}. The company is divided into two parts: a business and commercial section and a technical section.

This technical section consists of an hardware-oriented team with mechanical and electronics engineers, an embedded sofware team of low-level developers and a vision team that I have joined. This team is composed of Filip the Computer Vision lead and my personnal supervisor, Loic a Computer Vision engineer, Nima a Robotics engineer and Camille a Computer Vision trainee.

The objective of \groupname{} is to develop and sell an \emph{autonomous} and \emph{wireless} underwater drone called \iBubble.

\begin{figure}[h]
\centering
\includegraphics[width=0.4\textwidth]{iBubble3}
\caption{\iBubble{} drone}
\end{figure}


\iBubble{} is designed to replace cameraman divers. It can \emph{autonomously} follow a diver and capture footages from the diving session without any action from the user. The targeted \keyword{BtC} audience are for instance professional divers, diving centers, free-divers or resorts.

A live video feedback is also available on \iBubble. In this mode, a cable link between the drone and the user enables the video feedback on a smartphone and users can explore unknown spots before diving.\\


In addition to this \keyword{BtC} sector, \iBubble{} is suitable for a large range of \keyword{BtB} applications. Indeed, it can integrate multiple specific features for companies that have real needs in underwater inspection such as dam, harbor or simply ship's hull. Thanks to its \emph{autonomous} ability, it will be able to verify the good shape of a concrete wall for example and provides videos of its inspection.

%----------------------------------------------------------------------------------------

\section{Problematics}

As said in the previous section, \iBubble's main feature is to follow \emph{autonomously} a diver to film a video of the diving session. The diver can only choose one of the drone's available scenarii such as \emph{follow me}, \emph{circle me}, \emph{follow me side} etc. At any time the diver, or someone else is controlling the drone with a joystick. Because of the countless difficultis raised by an underwater environment (no GPS, no Wifi etc.), be able to follow a diver \emph{autonomously} is a technical challenge. To achieve this goal, \iBubble{} fits two technologies: \emph{acoustic} and \emph{vision} location.


\subsection{Acoustic System}

The diver is outfitted with a remote control that is an \emph{acoustic} emitter sending ultrasound pulses. \iBubble{} detects those pulses thanks to five \emph{acoustic} receivers and locates approximately the diver. This \emph{acoustic} positioning system is not very accurate because of the heavily perturbated underwater environment. As a consequence a \emph{vision} system is used to locate the diver in parallel.


\subsection{Vision System}

\iBubble{} has two cameras at his disposal: one to record the diving session (generally a \emph{GoPro} camera) and another one to capture frames used in the computer vision part. This camera is generally a cheaper camera with a low resolution. When \iBubble{} is opposite the diver it can detect the position more precisely thanks to several algorithms running on the embedded board (Machine learning detectors, tracking systems etc.).

\groupname{} chose \rasp\footnote{Website: https://www.raspberrypi.org/} as the central embedded computer for the drone. This embedded board is very cheap, reasonably powerful and very well-documented with a huge community of developers. Since all the vision system will be integrated on the \rasp{}, the difficulty is to achieve correct quality with a real-time processing on a such device.

The vision system is globally composed of two parts: a \emph{detector} to lock the position of the diver and a \emph{tracker} to follow him between two frames of the video. The bulk of complexity is in the computation of the \keyword{optical flow} for the tracking algorithm and in Forward Propagation of a Convolutional Neural Network for the detector part.

These algorithms are well-known and can be parallelize very efficiently. Therefore there are a lot of implementations on \keyword{GPU}s of these algorithms especially with the growth of \emph{Nvidia's CUDA cores} on embedded chip (Jetson TX2) for \emph{Internet of Thing} or \emph{autonomous} vehicles. However, there is no reliable implementation on embedded device such as the \rasp{} which is far less expensive than CUDA systems.

\rasp{} is fitted with the \vc{} designed by \emph{Broadcom}. Due to the growing popularity of the \rasp{}, \emph{Broadcom} released an official documentation (e.g. Appendice~\ref{AppendixA}) of the \vc{} architecture, but without any official \keyword{API}.

As a conseqence, the main questions for me is to study the architecture and understand how to use the \vc{}, then understand these algorithms and implement an optimized version of them on the \rasp{} with the help of a homemade \keyword{API} that use the \keyword{GPU} to perform some calculation in order to achieve a real time performance and to relieve the \keyword{CPU}.

%----------------------------------------------------------------------------------------

\section{Project and specifications}

Because of the wide range of subjects this problematic contains, from low-level programming to image processing and numerical analysis, the work has been split in two internships: another intern called Simon Bataille with a background on embedded system focused on General-Purpose \keyword{GPU} programming and me on the Computer Vision part.

Simon was in charge of the study of the Video Core IV \keyword{GPU} of the Rasperry Pi, how to program this Video Core, understand memory accesses, synchronization structures. The final step is to create an \keyword{api} for the \keyword{GPU} usable from a high-level language such as C++ containing basic functions running on the \keyword{GPU}.

My role is then to work on the \keyword{optical flow} algorithm for the tracking part and on a Machine Learning algorithm for the detector part. The vision system already exists and uses state-of-the-art algorithms such as Clustering of Static-Adaptive Correspondences for \keyword{cmt}\cite{Nebehay2015CVPR}, with the help of the main library used in Computer Vision: \keyword{OpenCV}\footnote{Website: https://opencv.org/}. But all the processing happens on the \keyword{CPU} because the Video Core of the Raspberry is not properly documented and there is no open source \keyword{API} to control it.

Then in order to implement these algorithm on the \keyword{GPU} it is primordial to have company's own implementation of these. It is my role to work on these algorithms with the technologies, libraries, frameworks that fit the best the problem. All the vision system is currently coded in C++ and use few libraries such \keyword{OpenCV} or Eigen\footnote{Website: https://eigen.tuxfamily.org/} library for linear algebra.

In a first time the focus is to study and understand the popular implementations of the \keyword{optical flow} algorithms, create a state-of-the-art document of these algorithms. Then suggest an optimized implementation for the chosen method and program it on the \keyword{GPU} with the help of Simon. Next this algorithm needs to be test in simulation and real conditions on the Raspberry Pi. This step is fundamental to estimate the gain and loss in quality and computing time compared to the current solution. Finally the last part is to implement it on the drone system and test the whole vision system with the new \keyword{optical flow} algorithm. This is the main subject for this internship.

In a second time, the problematic is to focus on Machine Learning algorithms to detect the diver in the image. Similarly, the goal is to suggest an optimized solution for the problem and then implement it on the Raspberry Pi and to test it in simulation and real conditions. This part of the project is complementary from the main subject about the optical flow, I will work on this part only after completion of the previous task.

Simon and I, we will work in parallel but at all time we will have to coordinate in order to create the dedicated functions on the GPU. I will have to specify the needs for the algorithms (what kind of computations such as a convolution function etc.) and Simon will have to adapt his \keyword{API} to integrate these functions on a program running of the \keyword{GPU}.

%----------------------------------------------------------------------------------------

\section{What this Template Includes}

\subsection{Folders}

This template comes as a single zip file that expands out to several files and folders. The folder names are mostly self-explanatory:

\keyword{Appendices} -- this is the folder where you put the appendices. Each appendix should go into its own separate \file{.tex} file. An example and template are included in the directory.

\keyword{Chapters} -- this is the folder where you put the thesis chapters. A thesis usually has about six chapters, though there is no hard rule on this. Each chapter should go in its own separate \file{.tex} file and they can be split as:
\begin{itemize}
\item Chapter 1: Introduction to the thesis topic
\item Chapter 2: Background information and theory
\item Chapter 3: (Laboratory) experimental setup
\item Chapter 4: Details of experiment 1
\item Chapter 5: Details of experiment 2
\item Chapter 6: Discussion of the experimental results
\item Chapter 7: Conclusion and future directions
\end{itemize}
This chapter layout is specialised for the experimental sciences, your discipline may be different.

\keyword{Figures} -- this folder contains all figures for the thesis. These are the final images that will go into the thesis document.

\subsection{Files}

Included are also several files, most of them are plain text and you can see their contents in a text editor. After initial compilation, you will see that more auxiliary files are created by \LaTeX{} or BibTeX and which you don't need to delete or worry about:

\keyword{example.bib} -- this is an important file that contains all the bibliographic information and references that you will be citing in the thesis for use with BibTeX. You can write it manually, but there are reference manager programs available that will create and manage it for you. Bibliographies in \LaTeX{} are a large subject and you may need to read about BibTeX before starting with this. Many modern reference managers will allow you to export your references in BibTeX format which greatly eases the amount of work you have to do.

\keyword{MastersDoctoralThesis.cls} -- this is an important file. It is the class file that tells \LaTeX{} how to format the thesis.

\keyword{main.pdf} -- this is your beautifully typeset thesis (in the PDF file format) created by \LaTeX{}. It is supplied in the PDF with the template and after you compile the template you should get an identical version.

\keyword{main.tex} -- this is an important file. This is the file that you tell \LaTeX{} to compile to produce your thesis as a PDF file. It contains the framework and constructs that tell \LaTeX{} how to layout the thesis. It is heavily commented so you can read exactly what each line of code does and why it is there. After you put your own information into the \emph{THESIS INFORMATION} block -- you have now started your thesis!

Files that are \emph{not} included, but are created by \LaTeX{} as auxiliary files include:

\keyword{main.aux} -- this is an auxiliary file generated by \LaTeX{}, if it is deleted \LaTeX{} simply regenerates it when you run the main \file{.tex} file.

\keyword{main.bbl} -- this is an auxiliary file generated by BibTeX, if it is deleted, BibTeX simply regenerates it when you run the \file{main.aux} file. Whereas the \file{.bib} file contains all the references you have, this \file{.bbl} file contains the references you have actually cited in the thesis and is used to build the bibliography section of the thesis.

\keyword{main.blg} -- this is an auxiliary file generated by BibTeX, if it is deleted BibTeX simply regenerates it when you run the main \file{.aux} file.

\keyword{main.lof} -- this is an auxiliary file generated by \LaTeX{}, if it is deleted \LaTeX{} simply regenerates it when you run the main \file{.tex} file. It tells \LaTeX{} how to build the \emph{List of Figures} section.

\keyword{main.log} -- this is an auxiliary file generated by \LaTeX{}, if it is deleted \LaTeX{} simply regenerates it when you run the main \file{.tex} file. It contains messages from \LaTeX{}, if you receive errors and warnings from \LaTeX{}, they will be in this \file{.log} file.

\keyword{main.lot} -- this is an auxiliary file generated by \LaTeX{}, if it is deleted \LaTeX{} simply regenerates it when you run the main \file{.tex} file. It tells \LaTeX{} how to build the \emph{List of Tables} section.

\keyword{main.out} -- this is an auxiliary file generated by \LaTeX{}, if it is deleted \LaTeX{} simply regenerates it when you run the main \file{.tex} file.

So from this long list, only the files with the \file{.bib}, \file{.cls} and \file{.tex} extensions are the most important ones. The other auxiliary files can be ignored or deleted as \LaTeX{} and BibTeX will regenerate them.

%----------------------------------------------------------------------------------------

\section{Filling in Your Information in the \file{main.tex} File}\label{FillingFile}

You will need to personalise the thesis template and make it your own by filling in your own information. This is done by editing the \file{main.tex} file in a text editor or your favourite LaTeX environment.

Open the file and scroll down to the third large block titled \emph{THESIS INFORMATION} where you can see the entries for \emph{University Name}, \emph{Department Name}, etc \ldots

Fill out the information about yourself, your group and institution. You can also insert web links, if you do, make sure you use the full URL, including the \code{http://} for this. If you don't want these to be linked, simply remove the \verb|\href{url}{name}| and only leave the name.

When you have done this, save the file and recompile \code{main.tex}. All the information you filled in should now be in the PDF, complete with web links. You can now begin your thesis proper!

%----------------------------------------------------------------------------------------

\section{The \code{main.tex} File Explained}

The \file{main.tex} file contains the structure of the thesis. There are plenty of written comments that explain what pages, sections and formatting the \LaTeX{} code is creating. Each major document element is divided into commented blocks with titles in all capitals to make it obvious what the following bit of code is doing. Initially there seems to be a lot of \LaTeX{} code, but this is all formatting, and it has all been taken care of so you don't have to do it.

Begin by checking that your information on the title page is correct. For the thesis declaration, your institution may insist on something different than the text given. If this is the case, just replace what you see with what is required in the \emph{DECLARATION PAGE} block.

Then comes a page which contains a funny quote. You can put your own, or quote your favourite scientist, author, person, and so on. Make sure to put the name of the person who you took the quote from.

Following this is the abstract page which summarises your work in a condensed way and can almost be used as a standalone document to describe what you have done. The text you write will cause the heading to move up so don't worry about running out of space.

Next come the acknowledgements. On this page, write about all the people who you wish to thank (not forgetting parents, partners and your advisor/supervisor).

The contents pages, list of figures and tables are all taken care of for you and do not need to be manually created or edited. The next set of pages are more likely to be optional and can be deleted since they are for a more technical thesis: insert a list of abbreviations you have used in the thesis, then a list of the physical constants and numbers you refer to and finally, a list of mathematical symbols used in any formulae. Making the effort to fill these tables means the reader has a one-stop place to refer to instead of searching the internet and references to try and find out what you meant by certain abbreviations or symbols.

The list of symbols is split into the Roman and Greek alphabets. Whereas the abbreviations and symbols ought to be listed in alphabetical order (and this is \emph{not} done automatically for you) the list of physical constants should be grouped into similar themes.

The next page contains a one line dedication. Who will you dedicate your thesis to?

Finally, there is the block where the chapters are included. Uncomment the lines (delete the \code{\%} character) as you write the chapters. Each chapter should be written in its own file and put into the \emph{Chapters} folder and named \file{Chapter1}, \file{Chapter2}, etc\ldots Similarly for the appendices, uncomment the lines as you need them. Each appendix should go into its own file and placed in the \emph{Appendices} folder.

After the preamble, chapters and appendices finally comes the bibliography. The bibliography style (called \option{authoryear}) is used for the bibliography and is a fully featured style that will even include links to where the referenced paper can be found online. Do not underestimate how grateful your reader will be to find that a reference to a paper is just a click away. Of course, this relies on you putting the URL information into the BibTeX file in the first place.

%----------------------------------------------------------------------------------------

\section{Thesis Features and Conventions}\label{ThesisConventions}

To get the best out of this template, there are a few conventions that you may want to follow.

One of the most important (and most difficult) things to keep track of in such a long document as a thesis is consistency. Using certain conventions and ways of doing things (such as using a Todo list) makes the job easier. Of course, all of these are optional and you can adopt your own method.

\subsection{Printing Format}

This thesis template is designed for double sided printing (i.e. content on the front and back of pages) as most theses are printed and bound this way. Switching to one sided printing is as simple as uncommenting the \option{oneside} option of the \code{documentclass} command at the top of the \file{main.tex} file. You may then wish to adjust the margins to suit specifications from your institution.

The headers for the pages contain the page number on the outer side (so it is easy to flick through to the page you want) and the chapter name on the inner side.

The text is set to 11 point by default with single line spacing, again, you can tune the text size and spacing should you want or need to using the options at the very start of \file{main.tex}. The spacing can be changed similarly by replacing the \option{singlespacing} with \option{onehalfspacing} or \option{doublespacing}.

\subsection{Using US Letter Paper}

The paper size used in the template is A4, which is the standard size in Europe. If you are using this thesis template elsewhere and particularly in the United States, then you may have to change the A4 paper size to the US Letter size. This can be done in the margins settings section in \file{main.tex}.

Due to the differences in the paper size, the resulting margins may be different to what you like or require (as it is common for institutions to dictate certain margin sizes). If this is the case, then the margin sizes can be tweaked by modifying the values in the same block as where you set the paper size. Now your document should be set up for US Letter paper size with suitable margins.

\subsection{References}

The \code{biblatex} package is used to format the bibliography and inserts references such as this one \parencite{Reference1}. The options used in the \file{main.tex} file mean that the in-text citations of references are formatted with the author(s) listed with the date of the publication. Multiple references are separated by semicolons (e.g. \parencite{Reference2, Reference1}) and references with more than three authors only show the first author with \emph{et al.} indicating there are more authors (e.g. \parencite{Reference3}). This is done automatically for you. To see how you use references, have a look at the \file{Chapter1.tex} source file. Many reference managers allow you to simply drag the reference into the document as you type.

Scientific references should come \emph{before} the punctuation mark if there is one (such as a comma or period). The same goes for footnotes\footnote{Such as this footnote, here down at the bottom of the page.}. You can change this but the most important thing is to keep the convention consistent throughout the thesis. Footnotes themselves should be full, descriptive sentences (beginning with a capital letter and ending with a full stop). The APA6 states: \enquote{Footnote numbers should be superscripted, [...], following any punctuation mark except a dash.} The Chicago manual of style states: \enquote{A note number should be placed at the end of a sentence or clause. The number follows any punctuation mark except the dash, which it precedes. It follows a closing parenthesis.}

The bibliography is typeset with references listed in alphabetical order by the first author's last name. This is similar to the APA referencing style. To see how \LaTeX{} typesets the bibliography, have a look at the very end of this document (or just click on the reference number links in in-text citations).

\subsubsection{A Note on bibtex}

The bibtex backend used in the template by default does not correctly handle unicode character encoding (i.e. "international" characters). You may see a warning about this in the compilation log and, if your references contain unicode characters, they may not show up correctly or at all. The solution to this is to use the biber backend instead of the outdated bibtex backend. This is done by finding this in \file{main.tex}: \option{backend=bibtex} and changing it to \option{backend=biber}. You will then need to delete all auxiliary BibTeX files and navigate to the template directory in your terminal (command prompt). Once there, simply type \code{biber main} and biber will compile your bibliography. You can then compile \file{main.tex} as normal and your bibliography will be updated. An alternative is to set up your LaTeX editor to compile with biber instead of bibtex, see \href{http://tex.stackexchange.com/questions/154751/biblatex-with-biber-configuring-my-editor-to-avoid-undefined-citations/}{here} for how to do this for various editors.

\subsection{Tables}

Tables are an important way of displaying your results, below is an example table which was generated with this code:

{\small
\begin{verbatim}
\begin{table}
\caption{The effects of treatments X and Y on the four groups studied.}
\label{tab:treatments}
\centering
\begin{tabular}{l l l}
\toprule
\tabhead{Groups} & \tabhead{Treatment X} & \tabhead{Treatment Y} \\
\midrule
1 & 0.2 & 0.8\\
2 & 0.17 & 0.7\\
3 & 0.24 & 0.75\\
4 & 0.68 & 0.3\\
\bottomrule\\
\end{tabular}
\end{table}
\end{verbatim}
}

\begin{table}
\caption{The effects of treatments X and Y on the four groups studied.}
\label{tab:treatments}
\centering
\begin{tabular}{l l l}
\toprule
\tabhead{Groups} & \tabhead{Treatment X} & \tabhead{Treatment Y} \\
\midrule
1 & 0.2 & 0.8\\
2 & 0.17 & 0.7\\
3 & 0.24 & 0.75\\
4 & 0.68 & 0.3\\
\bottomrule\\
\end{tabular}
\end{table}

You can reference tables with \verb|\ref{<label>}| where the label is defined within the table environment. See \file{Chapter1.tex} for an example of the label and citation (e.g. Table~\ref{tab:treatments}).

\subsection{Figures}

There will hopefully be many figures in your thesis (that should be placed in the \emph{Figures} folder). The way to insert figures into your thesis is to use a code template like this:
\begin{verbatim}
\begin{figure}
\centering
\includegraphics{Figures/Electron}
\decoRule
\caption[An Electron]{An electron (artist's impression).}
\label{fig:Electron}
\end{figure}
\end{verbatim}
Also look in the source file. Putting this code into the source file produces the picture of the electron that you can see in the figure below.

\begin{figure}[th]
\centering
\includegraphics{Figures/Electron}
\decoRule
\caption[An Electron]{An electron (artist's impression).}
\label{fig:Electron}
\end{figure}

Sometimes figures don't always appear where you write them in the source. The placement depends on how much space there is on the page for the figure. Sometimes there is not enough room to fit a figure directly where it should go (in relation to the text) and so \LaTeX{} puts it at the top of the next page. Positioning figures is the job of \LaTeX{} and so you should only worry about making them look good!

Figures usually should have captions just in case you need to refer to them (such as in Figure~\ref{fig:Electron}). The \verb|\caption| command contains two parts, the first part, inside the square brackets is the title that will appear in the \emph{List of Figures}, and so should be short. The second part in the curly brackets should contain the longer and more descriptive caption text.

The \verb|\decoRule| command is optional and simply puts an aesthetic horizontal line below the image. If you do this for one image, do it for all of them.

\LaTeX{} is capable of using images in pdf, jpg and png format.

\subsection{Typesetting mathematics}

If your thesis is going to contain heavy mathematical content, be sure that \LaTeX{} will make it look beautiful, even though it won't be able to solve the equations for you.

The \enquote{Not So Short Introduction to \LaTeX} (available on \href{http://www.ctan.org/tex-archive/info/lshort/english/lshort.pdf}{CTAN}) should tell you everything you need to know for most cases of typesetting mathematics. If you need more information, a much more thorough mathematical guide is available from the AMS called, \enquote{A Short Math Guide to \LaTeX} and can be downloaded from:
\url{ftp://ftp.ams.org/pub/tex/doc/amsmath/short-math-guide.pdf}

There are many different \LaTeX{} symbols to remember, luckily you can find the most common symbols in \href{http://ctan.org/pkg/comprehensive}{The Comprehensive \LaTeX~Symbol List}.

You can write an equation, which is automatically given an equation number by \LaTeX{} like this:
\begin{verbatim}
\begin{equation}
E = mc^{2}
\label{eqn:Einstein}
\end{equation}
\end{verbatim}

This will produce Einstein's famous energy-matter equivalence equation:
\begin{equation}
E = mc^{2}
\label{eqn:Einstein}
\end{equation}

All equations you write (which are not in the middle of paragraph text) are automatically given equation numbers by \LaTeX{}. If you don't want a particular equation numbered, use the unnumbered form:
\begin{verbatim}
\[ a^{2}=4 \]
\end{verbatim}

%----------------------------------------------------------------------------------------

\section{Sectioning and Subsectioning}

You should break your thesis up into nice, bite-sized sections and subsections. \LaTeX{} automatically builds a table of Contents by looking at all the \verb|\chapter{}|, \verb|\section{}|  and \verb|\subsection{}| commands you write in the source.

The Table of Contents should only list the sections to three (3) levels. A \verb|chapter{}| is level zero (0). A \verb|\section{}| is level one (1) and so a \verb|\subsection{}| is level two (2). In your thesis it is likely that you will even use a \verb|subsubsection{}|, which is level three (3). The depth to which the Table of Contents is formatted is set within \file{MastersDoctoralThesis.cls}. If you need this changed, you can do it in \file{main.tex}.

%----------------------------------------------------------------------------------------

\section{In Closing}

You have reached the end of this mini-guide. You can now rename or overwrite this pdf file and begin writing your own \file{Chapter1.tex} and the rest of your thesis. The easy work of setting up the structure and framework has been taken care of for you. It's now your job to fill it out!

Good luck and have lots of fun!

\begin{flushright}
Guide written by ---\\
Sunil Patel: \href{http://www.sunilpatel.co.uk}{www.sunilpatel.co.uk}\\
Vel: \href{http://www.LaTeXTemplates.com}{LaTeXTemplates.com}
\end{flushright}
