% Appendix E

\chapter{\file{driver.c} from \keyword{helloworld} directory} % Main appendix title
\definecolor{mGreen}{rgb}{0,0.6,0}
\definecolor{mGray}{rgb}{0.5,0.5,0.5}
\definecolor{mPurple}{rgb}{0.58,0,0.82}
\definecolor{backgroundColour}{rgb}{0.95,0.95,0.92}

%\lstdefinestyle{CStyle}{
%    backgroundcolor=\color{backgroundColour},
%    commentstyle=\color{mGreen},
%    keywordstyle=\color{magenta},
%    numberstyle=\tiny\color{mGray},
%    stringstyle=\color{mPurple},
%    %basicstyle=\scriptsize,
%    basicstyle=\tiny,
%    breakatwhitespace=false,
%    breaklines=true,
%    captionpos=b,
%    keepspaces=true,
%    numbers=left,
%    numbersep=5pt,
%    showspaces=false,
%    showstringspaces=false,
%    showtabs=false,
%    tabsize=2,
%    language=C
%}
\label{AppendixE} % For referencing this appendix elsewhere, use \ref{AppendixB}
%\lstinputlisting[style=CStyle]{driver.c}

\lstset{style=CStyle,caption={\file{driver.c} from \keyword{helloworld} directory}}

\begin{lstlisting}

#include <stdio.h>
#include <stdlib.h>
#include <string.h>
#include <stddef.h>
#include <sys/time.h>

#include "mailbox.h"
#include "qpu.h"

#define NUM_QPUS        1
#define MAX_CODE_SIZE   8192

static unsigned int qpu_code[MAX_CODE_SIZE];



struct memory_map
{
    unsigned int code[MAX_CODE_SIZE];
    unsigned int uniforms[NUM_QPUS][2];     // 2 parameters per QPU
                                            // first address is the input value
                                            // for the program to add to
                                            // second is the address of the
                                            // result buffer
    unsigned int msg[NUM_QPUS][2];
    unsigned int results[NUM_QPUS][16];     // result buffer for the QPU to
                                            // write into
};



int loadShaderCode(const char *fname, unsigned int *buffer, int len)
{
    FILE *in = fopen(fname, "r");
    if (!in)
    {
        fprintf(stderr, "Failed to open %s.\n", fname);
        exit(0);
    }

    size_t items = fread(buffer, sizeof(unsigned int), len, in);
    fclose(in);

    return items;
}



int main(int argc, char **argv)
{
    if (argc < 3)
    {
        fprintf(stderr, "Usage: %s <code .bin> <val>\n", argv[0]);
        return 0;
    }



    int code_words = loadShaderCode(argv[1], qpu_code, MAX_CODE_SIZE);
    printf("Loaded %d bytes of code from %s ...\n", code_words * sizeof(unsigned),
           argv[1]);



    struct GPU_FFT_HOST host;
    if (gpu_fft_get_host_info(&host))
    {
        fprintf(stderr,	"QPU fetch of host information (Rpi version, etc.) failed.\n");

	return -5;
    }



    unsigned uniform_val = atoi(argv[2]);
    printf("Uniform value = %d\n", uniform_val);



    volatile unsigned *peri = (volatile unsigned *) mapmem(host.peri_addr,
                              host.peri_size);
    if (!peri)
    {
        mem_free(mb, handle);
        qpu_enable(mb, 0);
        return -4;
    }



    int mb = mbox_open();
    if (qpu_enable(mb, 1))
    {
        fprintf(stderr, "QPU enable failed.\n");
        return -1;
    }
    printf("QPU enabled.\n");

    unsigned size = 1024 * 1024;
    unsigned handle = mem_alloc(mb, size, 4096, host.mem_flg);
    if (!handle)
    {
        fprintf(stderr, "Unable to allocate %d bytes of GPU memory", size);
        return -2;
    }

    unsigned ptr = mem_lock(mb, handle);



    void *arm_ptr = mapmem(BUS_TO_PHYS(ptr + host.mem_map), size);

    // assert arm_ptr ...
    struct memory_map *arm_map = (struct memory_map *)arm_ptr;
    memset(arm_map, 0x0, sizeof(struct memory_map));

    unsigned vc_uniforms = ptr + offsetof(struct memory_map, uniforms);
    unsigned vc_code = ptr + offsetof(struct memory_map, code);
    unsigned vc_msg = ptr + offsetof(struct memory_map, msg);
    unsigned vc_results = ptr + offsetof(struct memory_map, results);
    memcpy(arm_map->code, qpu_code, code_words * sizeof(unsigned int));



    for (int i = 0; i < NUM_QPUS; i++)
    {
        arm_map->uniforms[i][0] = uniform_val;
        arm_map->uniforms[i][1] = vc_results + i * sizeof(unsigned) * 16;
        arm_map->msg[i][0] = vc_uniforms + i * sizeof(unsigned) * 2;
        arm_map->msg[i][1] = vc_code;
    }

    unsigned ret = execute_qpu(mb, NUM_QPUS, vc_msg, GPU_FFT_NO_FLUSH,
                               GPU_FFT_TIMEOUT);



    // check the results!
    for (int i = 0; i < NUM_QPUS; i++)
    {
        for (int j = 0; j < 16; j++)
        {
            printf("QPU %d, word %d: 0x%08x\n", i, j, arm_map->results[i][j]);
        }
    }



    printf("Cleaning up.\n");
    unmapmem(arm_ptr, size);
    unmapmem((void *)host.peri_addr, host.peri_size);
    mem_unlock(mb, handle);
    mem_free(mb, handle);
    qpu_enable(mb, 0);
    printf("Done.\n");
}

\end{lstlisting}
